\chapter{Client API}
\label{client_api}

The Client API is used to access Web resources. It provides a higher-level API than \code{HttpURLConnection} as well as integration with JAX-RS providers. Unless otherwise stated, types presented in this chapter live in the \code{javax.ws.rs.client} package.

\section{Bootstrapping a Client Instance}

An instance of \Client\ is required to access a Web resource using the Client API. The default instance of \Client\ can be obtained by calling \code{newClient} on \ClientBuilder. \Client\ instances can be configured using methods inherited from \code{Configurable} as follows:

\begin{listing}{1}
// Default instance of client
Client client = ClientBuilder.newClient();

// Additional configuration of default client
client.property("MyProperty", "MyValue")
      .register(MyProvider.class)
      .register(MyFeature.class);
\end{listing}

See Chapter \ref{providers} for more information on providers. Properties are simply name-value pairs where the value is an arbitrary object.  Features are also providers and must implement the \code{Feature} interface; they are useful for grouping sets of properties and providers (including other features) that are logically related and must be enabled as a unit.

\section{Resource Access}
\label{resource_access}

A Web resource can be accessed using a fluent API in which method	 invocations are chained to build and ultimately submit an HTTP request. The following example gets a \code{text/plain} representation of the resource identified by \code{http://example.org/hello}:

\begin{listing}{1}
Client client = ClientBuilder.newClient();
Response res = client.target("http://example.org/hello")
    .request("text/plain").get();
\end{listing}

Conceptually, the steps required to submit a request are the following: (i) obtain an instance of \Client\ (ii) create a \WebTarget\ (iii) create a request from the \WebTarget\ and (iv) submit a request or get a prepared \Invocation\ for later submission. See Section \ref{invocations} for more information on using \Invocation.

Method chaining is not limited to the example shown above. A request can be further specified by setting headers, cookies, query parameters, etc. For example:

\begin{listing}{1}
Response res = client.target("http://example.org/hello")
    .queryParam("MyParam","...")
    .request("text/plain")
    .header("MyHeader", "...")
    .get();
\end{listing}

See the Javadoc for the classes in the \code{javax.ws.rs.client} package for more information.

\section{Client Targets}

The benefits of using a \WebTarget\ become apparent when building complex URIs, for example by extending base URIs with additional path segments or templates. The following example highlights these cases:

\begin{listing}{1}
WebTarget base = client.target("http://example.org/");
WebTarget hello = base.path("hello").path("{whom}");   
Response res = hello.resolveTemplate("whom", "world").request("...").get();
\end{listing}

Note the use of the URI template parameter~\code{\{whom\}}. The example above gets a representation for the resource identified by \code{http://example.org/hello/world}.

\WebTarget\ instances are {\em immutable} with respect to their URI (or URI template): methods for specifying additional path segments and parameters return a new instance of \WebTarget. However, \WebTarget\ instances are {\em mutable} with respect to their configuration. Thus, configuring a \WebTarget\ does not create new instances.

\begin{listing}{1}
// Create WebTarget instance base
WebTarget base = client.target("http://example.org/");
// Create new WebTarget instance hello and configure it
WebTarget hello = base.path("hello");
hello.register(MyProvider.class);
\end{listing}

In this example, two instances of \WebTarget\ are created. The instance \code{hello} inherits the configuration from \code{base} and it is further configured by registering \code{MyProvider.class}. Note that changes to \code{hello}'s configuration do not affect \code{base}, i.e.~inheritance performs a {\em deep} copy of the configuration. See Section \ref{configurable_types} for additional information on configurable types.

\section{Typed Entities}

The response to a request is not limited to be of type \Response. The following example upgrades the status of customer number 123 to ``gold status'' by first obtaining an entity of type \code{Customer} and then posting that entity to a different URI:

\begin{listing}{1}
Customer c = client.target("http://examples.org/customers/123")
    .request("application/xml").get(Customer.class);
String newId = client.target("http://examples.org/gold-customers/")
    .request().post(xml(c), String.class);
\end{listing}

Note the use of the {\em variant} \code{xml()} in the call to \code{post}. The class \code{javax.ws.rs.client.Entity} defines variants for the most popular media types used in JAX-RS applications.

In the example above, just like in the Server API, \jaxrs\ implementations are REQUIRED to use entity providers to map a representation of type \code{application/xml} to an instance of \code{Customer} and vice versa. See Section \ref{standard_entity_providers} for a list of entity providers that MUST be supported by all \jaxrs\ implementations.

\section{Invocations}
\label{invocations}

An invocation is a request that has been prepared and is ready for execution. Invocations provide a {\em generic interface} that enables a separation of concerns between the creator and the submitter. In particular, the submitter does not need to know how the invocation was prepared, but only how it should be executed: namely, synchronously or asynchronously.

 Let us consider the following example\footnote{The Collections class in this example is arbitrary and does not correspond to any specific implementation. There are a number of Java collection libraries available that provide this type of functionality.}:

\begin{listing}{1}
// Executed by the creator
Invocation inv1 = client.target("http://examples.org/atm/balance")
    .queryParam("card", "111122223333").queryParam("pin", "9876")
    .request("text/plain").buildGet();
Invocation inv2 = client.target("http://examples.org/atm/withdrawal")
    .queryParam("card", "111122223333").queryParam("pin", "9876")
    .request().buildPost(text("50.0"));
Collection<Invocation> invs = Arrays.asList(inv1, inv2);

// Executed by the submitter
Collection<Response> ress =
    Collections.transform(invs,
        new F<Invocation, Response>() {
             @Override
             public Response apply(Invocation inv) {
                 return inv.invoke(); } });
\end{listing}

In this example, two invocations are prepared and stored in a collection by the creator. The submitter then traverses the collection applying a transformation that maps an \Invocation\ to a \Response. The mapping calls \code{Invocation.invoke} to execute the invocation synchronously; asynchronous execution is also supported by calling \code{Invocation.submit}. Refer to
Chapter~\ref{asynchronous_processing} for more information on asynchronous invocations.

\section{Configurable Types}
\label{configurable_types}

The following Client API types are configurable: \Client, \ClientBuilder, and \WebTarget. Configuration methods are inherited from the \code{Configurable} interface implemented by all these classes. This interface supports configuration of:

\begin{description}
\item [Properties] Name-value pairs for additional configuration of features or other components of a \jaxrs\ implementation.
\item [Features] A special type of provider that implement the \code{Feature} interface and can be used to configure a \jaxrs\ implementation.
\item [Providers] Classes or instances of classes that implement one or more of the provider interfaces from Chapter \ref{providers}. A provider can be a message body reader, a filter, a context resolver, etc. 
\end{description}

The configuration defined on an instance of any of the aforementioned types is inherited by other instances created from it. For example, an instance of \WebTarget\ created from a \Client\ will inherit the \Client's configuration. However, any additional changes to the instance of \WebTarget\ will not impact the \Client's configuration and vice versa. Therefore, once a configuration is inherited it is also detached (deep copied) from its parent configuration and changes to the parent and child configurations are not be visible to each other.

\subsection{Filters and Entity Interceptors}
\label{filters_interceptors_client}

As explained in Chapter \ref{filters_and_interceptors}, filters and interceptors are defined as JAX-RS providers. Therefore, they can be registered in any of the configurable types listed in the previous section. The following example shows how to register filters and interceptors on instances of \Client\ and \WebTarget:

\begin{listing}{1}
// Create client and register logging filter
Client client = ClientBuilder.newClient().register(LoggingFilter.class);

// Executes logging filter from client and caching filter from target
WebTarget wt = client.target("http://examples.org/customers/123");
Customer c = wt.register(CachingFilter.class).request("application/xml")
    .get(Customer.class);
\end{listing}

In this example, \code{LoggingFilter} is inherited by each instance of \WebTarget\ created from \code{client}; an additional provider named \code{CachingFilter} is registered on the instance of \WebTarget.

\section{Reactive Clients}
\label{reactive_clients}

Section \ref{client_api_async} introduces asynchronous programming in the Client API. Asynchronous programming in \jaxrs\ enables clients to unblock certain threads by pushing work to background threads which can be monitored and possibly waited on (joined) at a later time. This can be accomplished in \jaxrs\ by either providing an instance of \code{InvocationCallback} or operating on the result of type \code{Future<T>} returned by an asynchronous invoker ---or some combination of both styles. 

Using \code{InvocationCallback} enables a more \emph{reactive} programming style in which user-provided code activates (or reacts) only when a certain event has occurred. Using callbacks works well for simple cases, but the source code becomes harder to understand when multiple events are in play. For example, when asynchronous invocations need to be composed, combined or in any way operated upon. These type of scenarios may result in callbacks that are nested inside other callbacks making the code far less readable ---often referred to as the \emph{pyramid of doom} because of the inherent nesting of calls.

To address the requirement of greater readability and to enable programmers to better reason about asynchronous computations, Java 8 introduces the a new interface called \code{CompletionStage} that includes a large number of methods dedicated to managing asynchronous computations. 

\jaxrs\ 2.1 defines a new type of invoker called \code{RxInvoker}, as well a default implementation of this type called \code{CompletionStageRxInvoker} that is based on the JDK 8 type \code{CompletionStage}. There is a new \code{rx} method which is used in a similar manner to \code{async} as described in \ref{client_api_async}. Let us consider the following example:

\begin{listing}{1}
CompletionStage<String> csf = client.target("forecast/{destination}")
    .resolveTemplate("destination", "mars")
    .request()
    .rx()
    .get(String.class);

csf.thenAccept(System.out::println);
\end{listing}

This example first creates an asynchronous computation of type \code{CompletionStage<String>}, and then simply waits for it to complete and displays its result (technically, a second computation of type \code{CompletionStage<Void>} is created on the last line simply to consume the result of the first computation). 

The value of \code{CompletionStage} becomes apparent when multiple asynchronous computations are necessary to accomplish a task. The following example obtains, in parallel, a price and a forecast for a destination and makes a reservation only if the desired conditions are met. 

\begin{listing}{1}
CompletionStage<Number> csp = client.target("price/{destination}")
    .resolveTemplate("destination", "mars")
    .request()
    .rx()
    .get(Number.class);

CompletionStage<String> csf = client.target("forecast/{destination}")
    .resolveTemplate("destination", "mars")
    .request()
    .rx()
    .get(String.class);
	
csp.thenCombine(csf, (price, forecast) -> 
    reserveIfAffordableAndWarm(price, forecast));
\end{listing}

Note that the \code{Consumer} passed in the call to method \code{thenCombine} requires the values of each stage to be available and, therefore, can only be executed after the two parallel stages are completed. 

As we shall see in the next section, support for \code{CompletionStage} is the \emph{default} for all \jaxrs\ implementations, but other reactive APIs may also be supported as extensions.

\subsection{Reactive API Extensions}
\label{reactive_api_extensions}

There have been several proposals for reactive APIs in Java. All \jaxrs\ implementations MUST support an invoker for \code{CompletionStage} as shown above. Additionally, \jaxrs\ implementations MAY support other reactive APIs using an extension built into the Client API. 

RxJava \cite{rxjava} is a popular reactive library available in Java. The type representing an asynchronous computation in this API is called an \code{Observable}. An implementation may support this type by providing a new invoker as shown in the following example:

\begin{listing}{1}
Client client = client.register(ObservableRxInvokerProvider.class);
	
Observable<String> of = client.target("forecast/{destination}")
    .resolveTemplate("destination", "mars")
    .request()
    .rx(ObservableRxInvoker.class)    // overrides default invoker
    .get(String.class);
	
of.subscribe(System.out::println);
\end{listing}

First, a provider for the new invoker must be registered on the \code{Client} object. Second, the type of the invoker must be specified as a parameter to the \code{rx} method. Note that because this is a \jaxrs\ extension, the actual names for the provider and the invoker in the example above are implementation dependent. The reader is referred to the documentation for the \jaxrs\ implementation of choice for more information.

Version 2.0 of RxJava \cite{rxjava2} has been completely re-written on top of the Reactive-Streams specification. This new architecture prompted the introduction of a new type called \code{Flowable}. JAX-RS implementations could easily support this new version by implementing a new provider (such as  \code{FlowableRxInvokerProvider}) and using the same pattern shown in the example above.

\section{Executor Services}
\label{executor_services}

Executor services can be used to submit asynchronous tasks for execution. \jaxrs\ applications can specify executor services while building a \code{Client} instance. Two methods are provided in \code{ClientBuilder} for this purpose, namely, \code{executorService} and \code{scheduledExecutorService}. 

In an environment that supports the Concurrency Utilities for Java EE \cite{concurrencyee}, such as the Java EE Full Profile, implementations MUST use \code{ManagedExecutorService} and \code{ManagedScheduledExecutorService}, respectively. The reader is referred to the Javadoc of \code{ClientBuilder} for more information about executor services.

