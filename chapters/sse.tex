\chapter{Server-Sent Events}
\label{sse}

\section{Introduction}
\label{sse_introduction}

Server-sent events (SSE) is a specification originally introduced as part of HTML5 by the W3C, but is currently maintained by the WHATWG \cite{sse}. It provides a way to establish a one-way channel from a server to a client. The connection is long running: it is re-used for multiple events sent from the server, yet it is still based on the HTTP protocol. Clients request the opening of an SSE connection by using the special media type \code{text/event-stream} in the \code{Accept} header.

Events are structured and contain several fields, namely, \code{event}, \code{data}, \code{id}, \code{retry} and \code{comment}. SSE is a messaging protocol where the \code{event} field corresponds to a topic, and where the \code{id} field can be used to validate event order and guarantee continuity. If a connection is interrupted for any reason, the \code{id} can be sent in a request header for a server to re-play past events ---although this is an optional behavior that may not be supported by all implementations. Event payloads are conveyed in the \code{data} field and must be in text format; \code{retry} is used to control re-connects and finally \code{comment} is a general purpose field that can also be used to keep connections alive.

\section{Client API}
\label{sse_client_api}

The \jaxrs\ client API for SSE was inspired by the corresponding JavaScript API in HTML5, but with changes that originate from the use of a different language. The entry point to the client API is the type \code{SseEventSource}, which provides a fluent builder similarly to other classes in the JAX-RS API. An \code{SseEventSource} is constructed from a \code{WebTarget} that is already configured with a resource location; \code{SseEventSource} does not duplicate any functionality in \code{WebTarget} and only adds the necessary logic for SSE. 

The following example shows how to open an SSE connection and read some messages for a little
while:

\begin{listing}{1}
  WebTarget target = client.target("http://...");
  try (SseEventSource source = SseEventSource.target(target).build()) {
	source.register(System.out::println);
	source.open();


	Thread.sleep(500); 		// Consume events for just 500 ms

  } catch (InterruptedException e) {

	// falls through

  }
\end{listing}

As seen in this example, an \code{SseEventSource} implements \code{AutoCloseable}. Before opening the source, the client registers an event consumer that simply prints each event. Additional handlers for other life-cycle events such as \code{onComplete} and \code{onError} are also supported, but for simplicity only \code{onEvent} is shown in the example above.

\section{Server API}
\label{sse_server_api}

The \jaxrs\ SSE server API is used to accept connections and send events to one or more clients. A resource method that injects an \code{SseEventSink} and produces the media type \code{text/event-stream} is an SSE resource method.

The following example accepts SSE connections and uses an executor thread to send 3 events before closing the connection:

\begin{listing}{1}
@GET

@Path("eventStream")

@Produces(MediaType.SERVER_SENT_EVENTS)

public void eventStream(@Context SseEventSink eventSink,

                        @Context Sse sse) {
  executor.execute(() -> {
    try (SseEventSink sink = eventSink) {


      eventSink.send(sse.newEvent("event1"));
      eventSink.send(sse.newEvent("event2"));
      eventSink.send(sse.newEvent("event3"));
	} 
  });
}
\end{listing}

SSE resource methods follow a similar pattern to those for asynchronous processing (see Section \ref{introduction_async}) in that the object representing the incoming connection, in this case \code{SseEventSink}, is injected into the resource method. 

The example above also injects the \code{Sse} type which provides factory methods for events and broadcasters. See section \ref{sse_broadcasting} for more information about broadcasting. Note that, just like \code{SseEventSource}, the interface \code{SseEventSink} is also auto-closeable, hence the use of the \emph{try-with-resources} statement above.

Method \code{send} on \code{SseEventSink} returns a \code{CompletionStage<?>} as a way to provide a \emph{handle} to the action of asynchronously sending a message to a client.

\section{Broadcasting}
\label{sse_broadcasting}

Applications may need to send events to multiple clients simultaneously. This action is called \emph{broadcasting} in \jaxrs. Multiple \code{SseEventSink}'s can be registered on a single \code{SseBroadcaster}. 

A broadcaster can only be created by calling method \code{newBroadcaster} on the injected \code{Sse} instance. The life-cycle and scope of an \code{SseBroadcaster} is fully controlled by applications and not the \jaxrs\ runtime. The following example shows the use of broadcasters, note the \code{@Singleton} annotation on the resource class:

\begin{listing}{1}
@Path("/")

@Singleton
public class SseResource {
	
  @Context
  private Sse sse;
  
  private volatile SseBroadcaster sseBroadcaster;
	
  @PostConstruct
  public init() {
    this.sseBroadcaster = sse.newBroadcaster();
  }


	
  @GET
  @Path("register")
  @Produces(MediaType.SERVER_SENT_EVENTS)
  public void register(@Context SseEventSink eventSink) {
    eventSink.send(sse.newEvent("welcome!"));
    sseBroadcaster.register(eventSink);
  }


	
  @POST
  @Path("broadcast")
  @Consumes(MediaType.MULTIPART_FORM_DATA)
  public void broadcast(@FormParam("event") String event) {

    sseBroadcaster.broadcast(sse.newEvent(event));
  }

}
\end{listing}

The \code{register} method on a broadcaster is used to add a new \code{SseEventSink}; the
\code{broadcast} method is used to send an SSE event to all registered consumers.

\section{Environment}
\label{sse_environment}

The \code{SseEventSource} class uses the existing \jaxrs\ mechanism based on \code{RuntimeDelegate} to find an implementation using the service name \code{javax.ws.rs.sse.SseEventSource.Builder}. The majority of types in the \code{javax.ws.rs.sse} are thread safe; the reader is referred to the Javadoc for more information on thread safety.


